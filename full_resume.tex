\documentclass{resume}

% Setup variables
\setname{Michael Koger Darden}
\setemail{mkogerd@gmail.com}
\setphone{(972) 762-6663}
\setwebsite{https://www.mkogerd.com}{www.mkogerd.com}

% Resume content
\begin{document}

\resumetitle

\section{Education}
\begin{center}
    \textbf{Bachelor of Science, Electrical Engineering, August 2018} \\
    The University of Texas at Austin \\
    \textbf{GPA: 3.76/4.00} \\
    \vspace{1em}
    \textbf{Related Courses} \\
    Software Design and Implementation (I \& II), Algorithms (\& Data Structures), Embedded Systems, Circuit Theory, Electromagnetic Engineering, Linear Systems \& Signals, Real-Time Digital Signal Processing, Digital Image \& Video Processing, Automatic Control, Probability, Principles of Data Science, Data Science Lab, Honors Senior Design
\end{center}

\section{Work experience}
\experience{Student Technician, UT LAITS}{Jun 2018 – Aug 2018}
\begin{itemize}
    \item Performed daily morning checks of classroom technical equipment
    \item Handled customer support calls and helped resolve technical issues quickly and efficiently
\end{itemize}
\experience{Student Technician, UT Applied Research Laboratories}{Jun 2016 – Aug 2016}
\begin{itemize}
    \item Automated LVS software toolchain installation and testing using Bash scripting to save time
    \item Setup GitLab Continuous Integration with automatic toolchain testing to increase efficiency
    \item Tested different tool versions using CI and git submodules to find stable updated tool versions
\end{itemize} 

\section{Academic Experience}
\experience{Honors Senior Design Project, The University of Texas at Austin}{Nov 2017 – May 2018}
\begin{itemize}
    \item Collaborated daily with a 5-member team to develop a team-formation web application for UT faculty
    \item Created a working algorithm prototype 2 months ahead of schedule in Python
    \item Before deadline, increased functionality while reducing runtime by 10x by restructuring Python algorithm
    \item Documented all steps of the design, research, and implementation process
\end{itemize}
\experience{Software Engineering and Design Lab, The University of Texas at Austin}{Jan 2018 – May 2018}
\begin{itemize}
    \item Co-developed a web-app using Python, HTML, and CSS to organize a database of 574 cocktail recipes
    \item Interfaced Google geocoding and timezone APIs to create a timezone-exchange Android app
    \item Co-developed an online blog site using Java and Google App Engine
\end{itemize}
\experience{Data Science Lab, The University of Texas at Austin}{Jan 2018 – May 2018}
\begin{itemize}
    \item Generated new Pokémon artwork with a convolutional GAN, Tensorflow, and Microsoft Azure
    \item Placed in top 33\% in a mock Kaggle competition by using XGBoost, data analysis, and feature engineering
\end{itemize}
\experience{Digital Image Processing Project, The University of Texas at Austin}{Nov 2017 – Dec 2017}
\begin{itemize}
    \item Implemented motion tracking on stationary videos to extract objects of interest
    \item Worked with MATLAB image processing and computer vision libraries
\end{itemize}
\experience{Principles of Data Science Project, The University of Texas at Austin}{Oct 2017 – Dec 2017}
\begin{itemize}
    \item Predicted outcomes of baseball games by using rolling averages of player statistics during a season
    \item Achieved an average accuracy higher than the home-team baseline by ensembling models from scikit-learn
\end{itemize}
\experience{Real-Time DSP Lab, The University of Texas at Austin}{Jan 2017 – May 2017}
\begin{itemize}
    \item Designed and implemented digital FIR and IIR filters
    \item Simulated software-defined radio and Implemented PAM transceivers
    \item Worked with signal generators, oscilloscopes, MATLAB, and TI Code Composer Studio
\end{itemize}
\experience{Software Design Project, The University of Texas at Austin}{Jun 2016 – Aug 2016}
\begin{itemize}
    \item Created a graphical critter simulator using Java
    \item Learned how to use java Reflection and JavaFX libraries as well as Scene Builder
\end{itemize}
\experience{Embedded Systems Project, The University of Texas at Austin}{Apr 2015 – May 2015}
\begin{itemize}
    \item Created a “tag” video game on the TM4C123 microcontroller using C and assembly, ranked as “supreme”
\end{itemize}
\experience{Robotathon 2015, UT Robotics and Automation Society}{Oct 2015 – Nov 2015}
\begin{itemize}
    \item Created a robot car to play RAS-ball
    \item Programmed in VIM
\end{itemize}
\experience{Robot Car Project, The University of Texas at Austin}{Oct 2014 – Dec 2014}
\begin{itemize}
    \item Programmed in Labview and built breadboard circuits
    \item Interfaced photoresistors and IR sensors
\end{itemize}

\section{Personal Projects}
\experience{Macro-tracker Web-App (\href{http://macros.mkogerd.com}{http://macros.mkogerd.com})}{Oct 2018}
\begin{itemize}
    \item Created a web-app for tracking macro nutrition using React.js, Node.js, and MySQL
    \item Designed an API that securely handles user authentication and database interactions using JWTs
\end{itemize}
\experience{Gravity IO Game (\href{http://game.mkogerd.com}{http://game.mkogerd.com})}{Jul 2018}
\begin{itemize}
    \item Launched an online multiplayer IO game made using Node.js, socket.io, HTML5, and ES6
    \item Implemented collision and gravity physics in Javascript as well as real-time player interactions and chat
\end{itemize}
\experience{HackTX 2017, The University of Texas at Austin (\href{http://dance.mkogerd.com}{http://dance.mkogerd.com})}{Oct 2017}
\begin{itemize}
    \item Scraped a web-archive of over 1000 dance videos to organize video meta-data into a CSV database
    \item Downloaded and reformatted videos using Python to increase video load-time and reduce size by 87.5%
    \item Improved video accessibility by creating a new dynamic front end using Python and Flask page templates
\end{itemize}
\experience{Personal Server}{Aug 2016}
\begin{itemize}
    \item Setup a Proxmox server to host chat, game, and web servers
    \item Setup Linux containers, VMs, and SSH with RSA encryption
\end{itemize}
\experience{3D design}{}
\begin{itemize}
    \item Designed an infinity-standing-desk using SOLIDWORKS and Git \hfill Jul 2016
    \item Designed and 3D printed a formicarium using SOLIDWORKS and MakerBot	\hfill Apr 2016
\end{itemize}
\experience{Arduino Projects}{}
\begin{itemize}
    \item Created an internet controllable desk-light using Javascript and PHP	\hfill Sept 2016
    \item Assembled a Bike-Wheel Display using Image Processing \hfill Feb 2016
    \item Created a 5V DC power supply \hfill Dec 2015
\end{itemize}
\experience{HackTX 2015, The University of Texas at Austin}{Sept 2015}
\begin{itemize}
    \item Created a static website using HTML/CSS
    \item Worked with GitHub
\end{itemize}

\section{Skills}
\textbf{Languages}: Java, C, C++, Python, Javascript, HTML, CSS, Bash, PHP, SQL, MATLAB, Android \\
\textbf{Tools}: Git, Flask, Node.js, Bootstrap, React.js, Google App Engine, Android, scikit-learn, TensorFlow, Labview, TI Code Composer Studio \\
\textbf{Other}: Windows, macOS, Linux, basic Portuguese, basic Spanish, limited Mandarin Chinese \\
Experience with soldering and breadboard-circuits \\
Experience with Arduino, Launchpad microprocessors, and TI TMS320C6700 Digital Signal Processors \\
Experience with GIMP and Photoshop \\
Experience with SOLIDWORKS 3D design \\

\section{Accomplishments}
\accomplishment{Texas Tricking Club, President}{Aug 2017 – May 2018}
\accomplishment{UT Social Dance, Class Assistant}{Aug 2016 – May 2018}
\accomplishment{Huawei Seeds for the Future, Participant}{Jul 2017}
\begin{itemize}[topsep=0pt]
    \item Selected as one of 18 participants nationwide to receive ICT training at Huawei HQ in Shenzhen, China
\end{itemize}
\accomplishment{Volunteer: English teacher in Peru}{2014}
\accomplishment{Volunteer: Veterinary assistant at Cape Town SPCA}{2013}
\accomplishment{Drumline Lieutenant, Outstanding Leadership Award}{2012-2013}
\accomplishment{Eagle Scout}{2009}

\end{document}